\documentclass[11pt]{article}

\usepackage{url}
\usepackage{graphicx}
%\usepackage{float}
\usepackage{upquote}
\setlength{\parskip}{0.5cm plus4mm minus3mm}

\textwidth=6.4in
\textheight=8.5in
\hoffset=-0.7in
\voffset=-0.7in

\setlength{\parindent}{0cm} 

\newcommand{\Yfun}{Y}

%\include{notation}

\hyphenation{Text-Wrangler}

\title{Chapter 1: Introduction to internal magnetic field inversion\\
  using altitude-cognizant gradient Vector Slepian functions}
\author{The Slepian Working Group}

\begin{document}
\maketitle


\section{Testing the software}

After installing the software and starting Octave or Matlab, and
initializing (by running \verb+initialize_octave+ or
\verb+initialize+), try out the demo function
\verb+demos_slepian_hotel+. For example. run

\verb+demos_slepian_hotel(1)+

This will take a while to calculate. If it takes too long you can
abort by pressing ``Ctrl-c''. Let's have a look why it takes so
long. Open the function \verb+demos_slepian_hotel.m+ by double-clicking
on it in Octave/Matlab.

You will first see a couple of lines with \verb+%+ signs in the
beginning. This is the help component of the program. Any later line
with a \verb+%+ in the beginning is simply a comment that does not do
anything but can be very helpful for anyone who reads the program.

You see that a few lines down there is a comment that says ``Maximum
spherical-harmonic degree:''. Underneath this comment is the line
``Lmax=20''. Here, Lmax defines the maximum spherical-harmonic
degree. I will give a more detailed explanation about spherical
harmonics at a later point but basically: The maximum
spherical-harmonic degree defines the resolution of features in our
analysis. The larger it is, the more details we see, but also the
longer the computations take and the more sensitive we are to noise in
the data!

Set Lmax to a lower value. Maybe 10. Save, and then run again

\qquad \verb+demos_slepian_hotel(1)+

This should, after other things, finally plot a sphere with some
blurry colors on it. You can rotate the sphere by clicking on ``R'' in
Octave, or the rotate button in Matlab and then rotate with your mouse.

If you now run the example again

\qquad \verb+demos_slepian_hotel(1)+

you will see that the calculations are much faster. This is because
in-between steps were stored.

Congrats, you calculated and plotted your first ``altitude-cognizant
gradient vector Slepian function``. Play around with the other
examples. You might need to set the maximum degree to a lower value to
make calculations faster, or to a higher value if you want a more
realistic and challenging example.

Also, in examples.m a few lines above Lmax you can read ``\% Region'',
below that \verb+%dom=[30.5 7.3];+ and below that \verb+%clon=18;+ and
\verb+ccola=90+33+. Remove the \verb+%+ on each of these lines and
instead type a \verb+%+ before \verb+dom='africa'+ and
\verb+clon=[]; ccola=[];+.

With this you commented (made invisible to the program) the line
setting Africa as domain and instead set a spherical ring between
opening angles~$7.3^\circ$ and~$30.5^\circ$ centered at
longitude~$18^\circ$ and latitude~$33^\circ$ (which is given as
colatitude $123^\circ$).

You can try other named regions such as 'namerica' for North America,
or 'samerica', or 'eurasia'. Or you can choose a polar caps for
example of opening angle 10 degrees by setting ``dom=10'' or a polar
ring by giving it two numbers like ``dom=[20 10];''. It is also
possible to run calculations for spherical cap centered somewhere else
than the north pole. See the help comments for \verb+glmalphaup.m+,
\verb+gradvecglmalphaup.m+, etc. You can even write a wrapper function
for a list of lon/lat locations of the outline of your generic
region. Generally, spherical caps and rings are much faster to
calculate and require much less memory than irregular regions. This is
particularly important for very large spherical-harmonic degrees (200
and beyond).

\verb+demos_slepian_hotel(3)+ and \verb+demos_slepian_hotel(4)+ are inversions for
artificial data with a noise level you can choose. These are the key
examples for estimating crustal fields from satellite altitude. In
principle you can take these examples (3 for radial component data
only and 4 for three component vectorial data), put in your data,
select your region, satellite altitude, planet radius, maximum
spherical harmonic degree, and number of Slepian functions you want to
use, and run it to calculate your crustal magnetic field model.

\verb+demos_slepian_hotel(5)+ finally puts everything together and
more. In this function we have:

\begin{itemize}
  \item An internal field (for example a
    magnetic field from a planet's crust or core),
  \item an external field (for example solar wind)
\end{itemize}
and we want to find the field within an spherical ring centered at a
chosen location and we know that the internal field only has power
within the spherical harmonic degrees of a chosen \emph{bandwidth},
for example spherical-harmonic degrees 10 to 40.



\end{document}
