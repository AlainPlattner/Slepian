\documentclass[11pt]{article}

\usepackage{url}
\usepackage{hyperref}
\usepackage{graphicx}
\usepackage{verbatim}
\usepackage{color}
\setlength{\parskip}{0.5cm plus4mm minus3mm}

\textwidth=6.4in
\textheight=8.5in
\hoffset=-0.7in
\voffset=-0.7in

\setlength{\parindent}{0cm} 

\newcommand{\Yfun}{Y}
%\newcommand{\TAG}{test}
\newcommand{\TAG}{\begin{color}{blue}This tutorial is currently under construction. Please check back later for more by keeping your software updated.\end{color}}


\hyphenation{Text-Wrangler}

\title{Chapter 2: Introduction to Slepian Functions}
\author{The Slepian Working Group}

\begin{document}
\maketitle

After having worked through Chapter 1: Introduction to Spherical
Harmonics, we will now have a look at what spherical Slepian functions
are. The goal of this chapter is not to give a solid foundation of the
mathematical intricacies. This can be obtained from the many research
articles in the literature. Rather, we will look at some basic ideas
and plot the first few Slepian functions.


\section{Basic ideas}
When plotting spherical harmonics in chapter 1, it became clear that
all of these spherical harmonics are functions that cover the entire
sphere. This is of course an advantage if we try to describe functions
(or maps), that cover the entire planet, such as gravity fields,
magnetic fields, etc. We saw in chapter 1 that we can generate pretty
much any spatial pattern we like by simply summing up different
spherical harmonics each multiplied by a different factor called
``coefficient''.

But what can we do if we only have information within a specific
region? Sometimes it's fine to just describe that information as
point values. But in some cases, as for example when we measure
gravity or magnetic data at satellite altitude, and we want to
calculate, what the corresponding magnetic or gravity field looks like
on the planet's surface (assuming there are no gravity or magnetic
sources between the planet and the satellite, or that they are
negligible), we need to describe these fields in spherical harmonics.

This is where the Slepian functions come in. Slepian functions are
linear combinations of spherical harmonics which means that they are
constructed by multiplying each spherical harmonic function with a
factor (called Slepian coefficient) and then add them all up. The name
Slepian function stems from the first author of a research article,
that described the original idea for 1-dimensional functions.

There are different types of Slepian functions that are constructed in
different ways but the basic idea is always to solve an optimization
problem, in which we try to balance the number of spherical-harmonic
functions we use (the maximum spherical-harmonic degree) in the
construction of the Slepian functions, and how much they are
concentrated within the region that we are interested in.



\TAG


\end{document}
