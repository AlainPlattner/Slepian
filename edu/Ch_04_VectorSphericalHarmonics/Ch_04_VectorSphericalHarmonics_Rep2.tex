\documentclass[11pt]{article}

\usepackage{url}
\usepackage{hyperref}
\usepackage{graphicx}
\usepackage{verbatim}
\usepackage{color}
\setlength{\parskip}{0.5cm plus4mm minus3mm}
\usepackage{upquote}
\usepackage{float}

\textwidth=6.4in
\textheight=8.5in
\hoffset=-0.7in
\voffset=-0.7in

\setlength{\parindent}{0cm} 

\newcommand{\Yfun}{Y}
%\newcommand{\TAG}{test}
\newcommand{\TAG}{\begin{color}{blue}This tutorial is currently under construction. Please check back later for more by keeping your software updated.\end{color}}

\newcommand{\HERE}{\begin{color}{blue}Currently working on this part.\end{color}}

\hyphenation{Text-Wrangler}

\title{Chapter 4: Vector Spherical Harmonics}
\author{Kylee Ford, Sarah Kroeker, Alain Plattner}

\begin{document}
\maketitle

\section{Representation 2 of Vector Fields}



$E_{lm}$: vector components from the gradient of a potential field from a planet. \\
$F_{lm}$: vector components from the gradient of a potential field from outside the satellite radius (space). \\
$C_{lm}$: same as in representation 1.

To evaluate the spherical harmonic coefficients, we must first convert each of the vector components into lmcosi format.  To do so, we can use the following:

\verb|elmcosi = ??|\\
\verb|flmcosi = fcoef2flmcosi(?coef=G(:,1)?,1);|\\
\verb|[blmcosi,clmcosi] = coef2blmclm(coef,L);|

Now we can convert these to xyz coordinates by running:

\verb|elm = elm2xyz(elmcosi,1);|\\
\verb|flm = flm2xyz(flmcosi,1);|\\
\verb|[blm,clm] = blmclm2xyz(blmcosi,clmcosi,1);|





\TAG
\end{document}